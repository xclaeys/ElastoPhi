% !TEX spellcheck = en_US

% !TEX root = elastophi-report.tex

\section*{Conclusion and outlook}

During our stay in the CEMRACS summer school, we have been able to implement a HM-ACA code and test it on several matrices provided by IFPEN. This way, we have learned a lot about the algorithm in itself and the details of its implementation. We hope that the obtained results will show the efficiency of the method for IFPEN applications, and in particular how we can obtain a great compression rate up with a good error of approximation.

\bigskip

We think that there are lots of tracks to keep on working and improving our code. The first one will be to add a way to not load the matrices and populate only the needed rows and columns for the ACA algorithm on the fly. This will allow the code to use the advantage of the algorithm in terms of storage because, then, we never need to assemble the entire dense matrix. Another improvement would be a better handling of the link between geometric elements and unknowns. Indeed for the moment we can just handle an arbitrary number of unknowns at one geometric point but we would like to be able to treat more general finite elements. Finally, one could think about parallelization for the matrix vector product with hierarchical matrices (see \cite[Section 3.1]{Bebendorf2008}) or for building them (see \cite[Section 3.4.6]{Bebendorf2008}).

At a more theoretical level, one could think of another admissibility criterion more adapted to a fracture crack network, for example taking into account the direction of the fractures using ellipsoids instead of balls. But it may be necessary to conduct a mathematical analysis of this problem to really find the right geometrical criterion.