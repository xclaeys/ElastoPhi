\section*{Introduction}

This document aims at presenting the work achieved for project Elasto$\Phi$ during the CEMRACS summer school  
that took place at the Centre de Rencontres Mathématiques (CIRM) at Luminy from  July the 24th to 
August the 25th, 2016. This project originated from mathematical challenges encountered by IFP Energies Nouvelles (IFPEN)
that was interested in solving an elastostatic problem in an infinite homogeneous background medium containing crack 
network with highly complex geometrical structure, see the pictures below. 

\begin{figure}[h!]
%\centerline{\includegraphics[width=0.4\linewidth]{../images/fig1.pdf}}
\end{figure}


\noindent 
This problem was reformulated as boundary integral equation posed at the surface of cracks, and 
discretized by means of Galerkin procedure based on piecewise constant functions, which is commonly known as 
Boundary Element Method (BEM). The fully non-local structure of boundary integral operators leads to  
densly populated matrices. As a consequence if the matrix of the problem is of size $N$, 
then any matrix-vector product (which is the most elementary building block of any iterative linear solver)
requires at least $\mathcal{O}(N^{2})$ operations.

\quad\\
As in most industrial applications, for the problems considered by IFPEN this situation is not acceptable: 
computational complexity of $N^{2}$ makes it impossible to perform a matrix-vector product for $N$ larger than
say $10^{6}$: this would be too costly in terms of time and memory storage. At the same time, the  applications 
considered by IFPEN typically require $N$ to be of this order.

\quad\\
To deal with this difficulty, IFPEN developped  an heuristic method consisting in forcing coefficients of the 
BEM matrix to zero whenever this coefficient corresponds to the interaction between sufficiently distant points 
of the crack network. With this procedure, the matrix of the problem is approximated by a sparsified matrix that 
allows matrix-vector products with $\mathcal{O}(N)$ complexity. But this strategy also induce substantial consistency 
error: measured in Frobenius norm, the perturbation on the matrix is typically  40\% large.

\quad\\
On the other hand, current literature on boundary integral equation nowadays offers a panel of refined 
complexity reduction techniques: fast multipole methods, hierarchical matrix strategies, and the like.   












