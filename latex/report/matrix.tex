% !TEX spellcheck = en_US

% !TEX root = elastophi-report.tex


\section{The matrix under consideration}\label{sec:GalerkinMatrix}

We now shall give some more details about the matrices under consideration during the Elasto$\Phi$ project. 
As previously mentioned, these matrices stemmed from a Galerkin discretization of a boundary integral 
equation. Assume that $\Gamma$ is a surface embedded in $\R^{3}$ that is the union of a collection 
$\mathcal{T} = \{\tau_{j}\}_{j=1}^{N}$ of flat triangles and quadrangles
\[
\Gamma = \mathop{\bigcup}_{j=1}^{N}\tau_{j}\;,\quad\quad \tau_{j} = \textrm{triangle or quadrangle}.
\]
The elements $\tau_{j}$ might intersect each other, which makes $\Gamma$ a potentially very rough surface.
Let us briefly describe the bilinear form associated with the variational formulation under consideration for 
the Galerkin discretization. We need to introduce the Green kernel of the problem, that is given by the explicit formula 
\[
\mathscr{G}(\bx) = \frac{1}{8\pi E}\frac{1+\rho}{1-\rho}\biggl( \frac{3-4\rho}{\vert \bx\vert}\mrm{Id}  +\frac{\bx\cdot\bx^{T}}{\vert \bx\vert^{3}}\biggr).
\]
It is fundamental to observe, and to keep in mind, that this function is singular at $\bx = 0$. In this expression, $\rho$ 
refers to the Poisson ratio, and $E$ is the elasticity module. Next, for each element $\tau$, let  $\bn_{\tau}$ refer to a vector field 
normal to $\tau$, and for any vector field $\bu\colon\Gamma\to \R^{3}$, define the trace operator
\[
\begin{array}{l}
\dsp{ \mathscr{R}_{\tau}(\bu) = \lambda\bn_{\tau}\div(\bu)+2\mu\,(\bn_{\tau}\cdot\nabla)\bu + \mu\bn_{\tau}\times\curl(\bu), }\\[10pt]
\dsp{ \textrm{with}\quad \lambda := \frac{E\rho}{(1+\mu)(1-2\rho)},\quad \mu := \frac{E}{2(1+\rho)}}.
\end{array}
\]
The variational formulation associated with the problem under consideration typically consists in finding $\bu\in \mH^{1/2}_{00}(\Gamma)^{3}$ 
such that $a(\bu,\bv) = f(\bv)$ for all $\bv\in \mH^{1/2}_{00}(\Gamma)^{3}$, where $\mH^{1/2}_{00}(\Gamma):=\Pi_{\tau\in\mathcal{T}}\mH^{1/2}_{00}(\tau)^{3}$
and the bilinear form is given by
\begin{equation}\label{VariationnalFormulation}
\begin{array}{l}
\dsp{ a(\bu,\bv) = \sum_{\tau\in\mathcal{T}}\sum_{\tau'\in\mathcal{T}}\int_{\tau\times\tau'}\mathscr{R}_{\tau}^{\by}(\mathscr{R}_{\tau'}^{\bx}\mathscr{G}(\bx-\by)) \bu(\bx)\bv(\by) d\sigma(\bx) d\sigma(\by).  }
\end{array}
\end{equation}
In this formula the operator $\mathscr{R}_{\tau}^{\bx}$ is the operator $\mathscr{R}_{\tau}$ applied with respect the $\bx$ variable. The operator $\mathscr{R}_{\tau'}^{\by}$ 
is defined accordingly. Formula  (\ref{VariationnalFormulation}) is rather abstract. Another, more explicit expression of this operator can be obtained, 
see e.g.~\cite{??}, but this is pointless for the present report. The only important thing here is that the integral kernel coming into play in this integral 
operator is singular at $\bx = \by$ (which is possible only if $\tau\cap \tau'\neq \emptyset$), and it is regular otherwise. In particular, if $\tau$ and $\tau'$
are distant from each other, then the operator associated with $a(\;,\;)$ is regularizing, and it will induce matrices with exponentially decreasing singular values.
Now let us describe how the bilinear form (\ref{VariationnalFormulation}) is discretized. Each space $\mH^{1/2}_{00}(\tau)^{3}$ is approximated with a three dimensional 
space 
\[
\mH_{h}(\tau)\simeq \mH^{1/2}_{00}(\tau)^{3}\quad \mrm{dim}(\mH_{h}(\tau)) = 3\quad \textrm{for each}\;\;\tau\in \mathcal{T},
\] 
that is to say a constant function for each direction. Next the global variational space is obtained by a simple cartesian product $\mH_{h}(\Gamma) = \Pi_{\tau\in \mathcal{T}}\mH_{h}(\tau)$.
Finally the discrete variational formulation writes: find $\bu_{h}\in \mH_{h}(\Gamma)$ such that 
\begin{equation}\label{discreteVF}
a(\bu_{h},\bv_{h}) = f(\bv_{h})\quad \bv_{h}\in \mH_{h}(\Gamma).
\end{equation}
Following the classical Galerkin discretization procedure, a matrix is derived from (\ref{discreteVF}) simply by specifying a basis for 
$\mH_{h}(\Gamma)$. In the present case, such a basis is specified as follows. Denote by $\bpsi_{3j-2}(\bx), \bpsi_{3j-1}(\bx),\bpsi_{3j}(\bx)$
three vector fields generating $\mH_{h}(\tau_{j})$ (recall that $\mrm{dim}(\mH_{h}(\tau_{j}) )= 3$ by hypothesis). Then we have 
$$
\mH_{h}(\Gamma) = \mathop{\mrm{span}}_{j=1\dots 3N}\{\bpsi_{j}\}.
$$
Of course, here, each $\bpsi_{3j-q}$ with $q=0,1,2$ is regarded as a function defined on $\Gamma$ that is supported 
only on $\tau_{j}$. Then the matrices that we had to deal with during Project Elatso$\Phi$ are defined as 
$\mA = (\mA_{j,k})_{j,k=1\dots 3N}$ where 
\begin{equation}\label{GalerkinMatrix}
\mA_{j,k}:=a(\bpsi_{j},\bpsi_{k}),\quad j,k = 1\dots 3N.
\end{equation}


